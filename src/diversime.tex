\begin{subsecao}{DiversIME}

O DiversIME, coletivo LGBTQIA+ do IME, tem como objetivo construir uma rede de
apoio, afeto e troca de conhecimento, de modo a amparar e unificar a luta da
comunidade de IMEanos que representem a diversidade sexual, afetiva e de gênero. 

Tendo em vista nossos propósitos, organizamos reuniões periódicas onde trazemos
a debate temas importantes para nossa atuação como coletivo, por meio de
atividades que instiguem discussão e reflexão acerca dos tópicos levantados.
Além disso, desejamos que esses encontros possam servir como um espaço de
acolhimento para as pessoas LGBTQIA+ do IME. Acreditamos na importância da
existência de um ambiente seguro e receptivo para a nossa comunidade na
Universidade, para que sejamos capazes de expressar nossas individualidades
e trocar experiências sem medo de julgamentos. Além disso, estamos prontos
para defender aqueles que se sintam hostilizados e agir nessa situação. 

Sabemos bem que a vida durante a pandemia foi especialmente difícil, e que as 
consequências sociais dessa situação podem ter sido agravadas para diversos integrantes
de grupos marginalizados. Mas acredite, você não está só, e através do DiversIME
você pode encontrar pessoas que vão te entender e estar ao seu lado!

Sendo assim, entre em nossos grupos de Whatsapp!

Basta mandar uma mensagem pedindo que te adicionem, para nosso Instagram
(@divers$\_$ime) ou via contato direto com um dos nossos membros.

\end{subsecao}
