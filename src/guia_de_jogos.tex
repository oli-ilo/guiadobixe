\begin{secao}{Guia de jogos da Vivência}

Bixes, como já foi dito muitas vezes, na USP vocês não devem apenas estudar, mas
também aproveitar TUDO que é oferecido. Se você é alguém que gosta de jogar
baralho, no IME existem muites veteranes que ficarão felizes em te chamar para
jogar se estiverem precisando de mais um jogador.

A maior concentração dessus veteranes acontece na Vivência, e lá elus jogam
principalmente os seguintes jogos: Truco, Fodinha, Pokeralho, Cagando, Copas,
Espadas, King e Bridge. Como sabemos que vocês provavelmente nunca ouviram falar de
muitos desses jogos, tivemos a bondade de ensiná-los por meio desse pequeno manual
de jogos de baralho da Vivência, assim quando vocês tiverem tempo já estarão
prontes para jogar com os veteranes. Não batam de mico* e leiam-no com
atenção.\footnote{Os termos marcados com um * - e alguns não marcados - são
explicados no Glossário, no fim do Guia de Jogos. Vocês, bixes, provavelmente não
vão entender tudo o que está escrito aqui. Nesse caso, é só ir até a Vivência e
perguntar pra algume veterane.}

Vamos então aos jogos!

\input{jogos_truco.tex}
\input{jogos_fodinha.tex}
\input{jogos_pokeralho.tex}
\input{jogos_cagando.tex}
\input{jogos_copas.tex}
\input{jogos_espadas.tex}
\input{jogos_king.tex}
\input{jogos_bridge.tex}
\input{jogos_glossario.tex}

\end{secao}
