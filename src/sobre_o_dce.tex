\begin{secao}{Um Pouco Sobre o DCE}

O Diretório Central dos Estudantes Livre da USP ``Alexandre Vannucchi Leme'' é a
entidade que nos representa: estudantes de todos os campi e cursos da nossa universidade.
Entidade que tem sua história marcada pela defesa da educação pública, liberdade
de organização e atuação política, sobreviveu na clandestinidade por alguns anos
da ditadura militar, por proibição do regime.

No entanto, a ditadura não conseguiu acabar com o movimento estudantil. Contribuiu,
ao contrário, para que es estudantes percebessem seu papel importantíssimo como
protagonistas das mudanças que queriam. A partir dessa união, garantiu-se a
refundação do DCE da USP, em 1976, com caráter LIVRE, que representa a autonomia
des estudantes e a não vinculação às estruturas do Estado e da reitoria.

No ano de 1973, Alexandre Vannucchi Leme tinha 22 anos e cursava o quarto ano
de Geologia na USP. ``Minhoca'', como foi apelidado por seus amigos de curso, participava
do movimento estudantil e lutava pela democracia no país. Na manhã do dia 16 de março,
foi levado pelo exército, torturado e morto dois dias depois nos porões do DOI-CODI,
órgão responsável pela perseguição e repressão política na época, em São Paulo.

Em homenagem a ele e a todos os seus semelhantes, vítimas de repressão, o DCE recebe este nome.

Em tempos mais recentes, a necessidade pela defesa da USP pública, gratuita, de qualidade
e democrática se faz cada vez mais necessária e urgente. Para que nós consigamos
garantir essas reivindicações na USP, todes devem ser protagonistas dessa defesa,
por entendermos que é fundamental a existência de uma educação pública com qualidade
em um país marcado pela desigualdade.

Existem dois espaços principais do DCE que es estudantes precisam conhecer: A Assembléia 
Geral e o Conselho de Centros Acadêmicos.
A Assembléia Geral é o ambiente mais importante de deliberação dos estudantes da USP. Funciona de um
jeito parecido com uma assembléia de condomínio: es estudantes se reúnem num lugar grande e todes es
presentes tem direito a falar, os assuntos a serem discutidos são definidos previamente, e no final
são votados encaminhamentos sobre esse assunto. Todes es estudantes tem direito a votar. Participar das
Assembléias Gerais é muito importante para construção da democracia estudantil.
O Conselho de Centros Acadêmicos (CCA) é um pouco diferente: ele é aberto para participação de todes es alunes,
mas só os Centros Acadêmicos tem direito a voto (note: Centro Acadêmico é a representação des estudantes de
um curso ou instituto específico; no caso do IME, é o CAMat). Mesmo assim, é importante participar do CCA
para se inteirar dos debates e expressar sua opinião, além de acompanhar o que es nosses representantes 
estão fazendo nesse espaço. 

Para conhecer melhor o DCE da USP, confira o perfil no Instagram (\url{https://www.instagram.com/dceusp}).

\end{secao}
